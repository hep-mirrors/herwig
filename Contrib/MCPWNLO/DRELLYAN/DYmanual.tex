\documentclass[12pt,a4paper,oneside]{article}
%\documentclass{article}
%\usepackage{afterpage}
%\usepackage[hang,small,bf]{caption}
%\usepackage{fancyhdr}
%\usepackage[something]{optional}
%\usepackage[todo, question, colour]{optional}
%\usepackage{epsfig,axodraw}
%\usepackage{array,cite,amsmath,amssymb}
%\usepackage{cite}
%\usepackage{makeidx}
%\usepackage{doublespace}  % This doesn't work with FeynArts diagrams
%\usepackage{array}
%\usepackage{multicol}
%\usepackage{subfigure}
%\usepackage{rotating}
%\usepackage{amssymb}
\usepackage{setspace}
\usepackage{booktabs}
%\input{psfig.tex}
%\usepackage{psfig}
\oddsidemargin=-0.05in
\textwidth=6.9in
\topmargin=-0.8in
\textheight=9.55in
%\reqno
%\newcounter{table}
\addtocounter{table}{0}
\usepackage{hyperref}
\usepackage{epsfig,bm,amsmath}
\onehalfspacing
% \makeatletter
% \DeclareRobustCommand{\Cpp}
% {\valign{\vfil\hbox{##}\vfil\cr
%    \textsf{C\kern-.1em}\cr
%    $\hbox{\fontsize{\sf@size}{0}\textbf{+\kern-0.05em+}}$\cr}%
% }
\begin{document}
\begin{center}
\Large \textbf {Drell-Yan Vector Boson Production with {\tt MCPWNLO}} \\
\end{center}
\section{Introduction}
This manual describes how to use this program to generate Drell-Yan events at hadron colliders with NLO accuracy using the {\tt MC@NLO} \cite{Frixione:2002ik} and {\tt POWHEG} \cite{Nason:2004rx} methods. More details on the use and application of the program and its interface with {\tt Herwig++} \cite{Bahr:2008pv} can be found in \cite{LatundeDada:2007jg, seyi}.
\section{Setting the parameters}
Within the directory {\tt MCPWNLO/DRELLYAN}, the file {\tt DYPP\_INPUTS.h} includes all the available user parameters
for the main program {\tt DYPP.cxx}. The parameters are:\\
\\
{\tt bool MCNLO}: Set to {\tt true} if the {\tt MC@NLO} method is to be used or {\tt
  false} if the {\tt POWHEG} method is to be used. \\
{\tt bool trunc}: Set to {\tt true} if truncated shower of at most one gluon is to be
implemented for {\tt POWHEG} method. Note that to interface with an angular ordered
shower, this should be set to {\tt true}. \\
{\tt int VBID}: Particle ID for the vector boson required. Set to {\tt 23} for $Z$ boson
production or {\tt 24} for $W$ boson production. \\
{\tt double cme}: The hadron-hadron center of mass energy in GeV e.g. 1800/1960 for the
Tevatron and 14000 for the LHC. \\
{\tt bool acc}: Type of collider. Set to {\tt true} if $p-p$ or {\tt false} if
{$p-\bar{p}$. \\
{\tt double Mz}: Pole mass of the $Z$ boson in GeV.
{\tt double Mw}: Pole mass of the $W$ boson in GeV.
{\tt bool zerowidth }: Set to {\tt true} if zero width approximation is to be used or {\tt
  false} if Breit-Wigner resonance is to be used for boson mass. \\
{\tt double widthw}: Width of the $W$ boson in GeV.\\
{\tt double widthz}: Width of the $Z$ boson in GeV.\\
{\tt double halfwidth}: Number of halfwidths either side of the resonance peak. \\
{\tt char* rscheme}: Factorization scheme. Options are ``{\tt MSbar}'' and ``{\tt DIS}''.\\
{\tt char* PDFset}: The PDFset to be used e.g. ``{\tt cteq5m.LHgrid}''. Make sure this agrees with the factorization
scheme set above. \\
{\tt int nevint}: Number of events (typically $ \approx 10^5$) to integrate over for cross-section calculation and
determination of maximum weights. \\
{\tt int nevgen}: Number of events (typically $\approx 10^5$) to generate. \\
{\tt int rseed}:  Initial seed for the random number generator. \\ 

\section{Generating partonic events}
After setting the parameters, open the {\tt Makefile} and set {\tt LHAPDFDIR} to your
{\tt LHAPDF} directory. Also set {\tt HERWIGDIR} to the address of the {\tt Herwig} folder
in the directory {\tt MCPWNLO}. 
To run {\tt DYPP.cxx}, in the folder {\tt DRELLYAN}, type the following commands :\\
\\
{\tt make clean} \\
{\tt make} \\
\\
This creates the executable {\tt DYPP} and {\tt run\_dypp} (which is moved to {\tt
  HERWIGDIR}). Next, type: 
{\tt ./DYPP} \\
This runs the main program and generates the Les Houches file for interface with \textsf{Herwig++}. This file will be called {\tt PWDYPP.dat} if running {\tt POWHEG} and {\tt
  MCDYPP.dat} if running {\tt MC@NLO}. It contains unweighted events with absolute weights
of $1$.
\section{{\tt POWHEG} requirements}
If running {\tt POWHEG}, go to the folder {\tt PWInstallFiles} in the main {\tt MCPWNLO}
directory. There you will find the following files:\\
{\tt PartnerFinder.cc}  {\tt PartnerFinder.h}  {\tt PartnerFinder.icc}
Replace the files of the same names in {\tt /Shower/Base} folder of your \textsf{Herwig++}
installation directory. Then go back to the {\tt Shower} folder not in {\tt Base}!) and type: \\
\\
{\tt make} \\
{\tt make install} \\
\\
This allows us to set the colour partner of the hardest emission correctly for {\tt
  POWHEG}.
\section{Analysis} 

In the folder {\tt DYAnalysis} in the {\tt DRELLYAN} directory are some analysis files
which analyze the events after interfacing the Les Houches file with
\textsf{Herwig++}. \\
{\tt MySimpleAnalysis.cc} contains the main program which provides
histograms for the transverse momenta of the dilepton pair produced from the vector
bosons, the rapidity and azimuthal distributions of the bosons, the masses of the bosons and the rapidity and
azimuths of the leptons. Other histograms can be added by booking histograms in the {\tt
  MySimpleAnalysis.icc} file and declaring them in the {\tt MySimpleAnalysis.h}.\\
Open the {\tt Makefile} and set {\tt HPDIR} and
{\tt THEPEGDIR} to the folder where you installed \textsf{Herwig++} and
\textsf{ThePEG}. Compile the directory by typing the following commands. \\
\\
{\tt make} \\
{\tt make install} \\
\\
in the {\tt DYAnalysis} directory. This recreates the library, {\tt MySimpleAnalysis.so}.

\section{Interfacing with \textsf{Herwig++}}
Having generated the Les Houches file and set up the analysis handler, the next step is to
run \textsf{Herwig++}. It is assumed that both \textsf{Herwig++} and \textsf{ThePEG} have
already been installed on your system. 

Now go to the directory {\tt MCPWNLO/Herwig} and open the initialization file {\tt
  DYPP.in}. This contains \textsf{Herwig++} user parameters which can be set depending on
which process and NLO method you are running. The default set-up is for {\tt MC@NLO} $W$
boson production at the $1800$ GeV Tevatron. A selection of these are:\\
\\
{\tt set Reader2:FileName DYPP.dat}: This is the filename for the Les Houches file your
generated file is converted into by the program {\tt DRELLYAN/run\_dypp.cxx}. Leave this
as it is! \\
{\tt set Reader2:EBeamA 900.0}: The beam energy in GeV of the hadron from the `left'. \\
{\tt set Reader2:EBeamB 900.0}: The beam energy in GeV of the hadron from the `right'.\\
{\tt set LesHouchesHandler:WeightOption NegUnitWeight}: The weight option for the
events. This allows for negative weighted events. \\
{\tt insert SimpleQCD:MatrixElements[0] MEqq2W2ff}: The hard process. Here it's set up for
$W$ boson production.\\
\\
Next are commands which should be uncommented if running {\tt POWHEG}. If running {\\
tt MC@NLO}, comment these out. \\
\\
{\tt set /Herwig/Shower/Evolver:HardVetoMode 0}: The veto mode to be applied. For {\tt
  POWHEG}, this should be set to $1$.\\
{\tt set /Herwig/Shower/PartnerFinder:PHPartnerFinder 0}: The partner finder option. This
should be set to $1$ for {\tt POWHEG}.\\
{\tt set /Herwig/Shower/Evolver:HardVetoScaleSource 0}: This reads the maximum $p_T$ for
the veto from {\tt SCALUP} in the Les Houches file. Set this to $1$ for {\tt POWHEG}.\\
\\
Having set up the initialization file, run \textsf{Herwig++} by typing the following
commands: \\
\\
{\tt ./run\_dypp [eventfile] [number of events]} \\
\\
An example of the run
command is:\\
\\
{\tt ./run\_dypp /usera/seyi/MCPWNLO/DRELLYAN/MCDYPP.dat 10000}\\
\\
where you should replace the eventfile with the path to your Les Houches file.
\\
  At the end of the run, a topdraw file called {\tt MG-MySimpleAnalysis.top} will be
  produced containing the histograms booked by the analysis handler. If you have topdraw
  installed, you can convert this to a postscript file by typing the command:\\
\\
{\tt td -dPOSTSCRIPT  MG-MySimpleAnalysis.top}\\
\\
In the Analysis folder, the files ZpTRun1.top and WpTRun1.top contain Tevatron Run 1 data
which can be used for comparisons.
\section{Further Information}
For further information about {\tt MCPWNLO} contact: {\tt seyi@hep.phy.cam.ac.uk}.
\bibliography{manual}
\bibliographystyle{utphys}
\end{document}
%%% Local Variables: 
%%% mode: latex
%%% TeX-master: t
%%% End: 
