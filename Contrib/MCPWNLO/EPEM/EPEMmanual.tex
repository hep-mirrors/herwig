\documentclass[12pt,a4paper,oneside]{article}
%\documentclass{article}
%\usepackage{afterpage}
%\usepackage[hang,small,bf]{caption}
%\usepackage{fancyhdr}
%\usepackage[something]{optional}
%\usepackage[todo, question, colour]{optional}
%\usepackage{epsfig,axodraw}
%\usepackage{array,cite,amsmath,amssymb}
%\usepackage{cite}
%\usepackage{makeidx}
%\usepackage{doublespace}  % This doesn't work with FeynArts diagrams
%\usepackage{array}
%\usepackage{multicol}
%\usepackage{subfigure}
%\usepackage{rotating}
%\usepackage{amssymb}
\usepackage{setspace}
\usepackage{booktabs}
%\input{psfig.tex}
%\usepackage{psfig}
\oddsidemargin=-0.05in
\textwidth=6.9in
\topmargin=-0.8in
\textheight=9.55in
%\reqno
%\newcounter{table}
\addtocounter{table}{0}
\usepackage{hyperref}
\usepackage{epsfig,bm,amsmath}
\onehalfspacing
% \makeatletter
% \DeclareRobustCommand{\Cpp}
% {\valign{\vfil\hbox{##}\vfil\cr
%    \textsf{C\kern-.1em}\cr
%    $\hbox{\fontsize{\sf@size}{0}\textbf{+\kern-0.05em+}}$\cr}%
% }
\begin{document}
\begin{center}
\Large \textbf {LEP $e^+e^-$ annihilation with {\tt MCPWNLO}} \\
\end{center}
\section{Introduction}
This manual describes how to generate $e^+e^-$ annihilation events at LEP with NLO accuracy using the {\tt MC@NLO} \cite{Frixione:2002ik} and {\tt POWHEG} \cite{Nason:2004rx} methods. More details on the use and application of the program and its interface with {\tt Herwig++} \cite{Bahr:2008pv} can be found in \cite{LatundeDada:2007jg,LatundeDada:2006gx}.
\section{Setting the parameters}
Within the directory {\tt MCPWNLO/EPEM}, the files {\tt MCEPEM\_INPUTS.h} and {\tt PWEPEM\_INPUTS.h} include all the available user parameters
for the main programs {\tt MCEPEM.cxx} and {\tt PWEPEM.cxx} for {\tt MC@NLO} and {\tt
  POWHEG} event generation respectively. The {\tt MCEPEM} parameters are:\\
\\
{\tt double cme}: The center of mass energy in GeV e.g. $91.2$ for LEP.\\
{\tt double Mz}: Pole mass of the $Z$ boson in GeV.\\
{\tt int nf}: Number of parton flavours ($1-5$).\\
{\tt bool massiveME }: Set to {\tt true} if a non-zero parton mass is to be used in the matrix element or {\tt
  false} if all partons are to be considered massless.\\
{\tt double alphasmz} : The value of $\alpha_S$ at the renormalization scale e.g. Mz for
LEP. \\
{\tt bool boost}: Set to {\tt true} if masses should be boosted to true parton masses if
{\tt massiveME} above is set to {\tt false}.\\
{\tt int it}: Maximum number of iterations for the Newton-Raphson boost.\\
{\tt bool resolve}: Set to {\tt true} if the resolution cut on very soft emissions is to be implemented if {\tt massive}
is set to {\tt true}.
{\tt int nevint}: Number of events (typically $\approx 10^5$) to integrate over for cross-section calculation and
determination of maximum weights. Note that this is missing from {\tt PWEPEM\_INPUTS.h}. \\
{\tt int nevgen}: Number of events to generate (typically $ \approx 10^5$). \\
{\tt int rseed}:  Initial seed for the random number generator. \\ 
\\
In {\tt PWEPEM\_INPUTS.h}, there are the following additional parameters:\\
\\
{\tt double Lambda} : The value of $\Lambda_{\rm QCD}$ for the running of $\alpha_S$. \\
{\tt bool trunc}: Set to {\tt true} if the truncated shower is to be switched on.
\section{Generating partonic events}
After setting the parameters, open the {\tt Makefile} and set {\tt HERWIGDIR} to the address of the {\tt Herwig} folder
in the directory {\tt MCPWNLO}. 
To run {\tt MCEPEM.cxx} or {\tt PWEPEM.cxx}, in the directory {\tt EPEM}, type the following commands :\\
\\
{\tt make clean} \\
{\tt make} \\
\\
This creates the executables {\tt MCEPEM}, {\tt PWEPEM} and {\tt run\_epem} (which is moved to {\tt
  HERWIGDIR}). Next, type: \\
{\tt .$\backslash$MCEPEM} \\ 
or\\
{\tt .$\backslash$PWEPEM} \\ 
depending on which NLO method is being applied.
This runs the main program and generates the Les Houches file for interface with \textsf{Herwig++}. This file will be called {\tt PWEPEM.dat} if running {\tt POWHEG} and {\tt
  MCEPEM.dat} if running {\tt MC@NLO} containing unweighted events with absolute weights
of $1$.
\section{{\tt POWHEG} requirements}
If running {\tt POWHEG}, go to the folder {\tt PWInstallFiles} in the main {\tt MCPWNLO}
directory. There you will find the following files:\\
{\tt PartnerFinder.cc}  {\tt PartnerFinder.h}  {\tt PartnerFinder.icc}
Replace the files of the same names in {\tt /Shower/Base} folder of your \textsf{Herwig++}
installation directory. Then go back to the {\tt Shower} folder (not in {\tt Base}!) and type: \\
\\
{\tt make} \\
{\tt make install} \\
\\
This allows us to set the colour partner of the hardest emission correctly for {\tt
  POWHEG}.
\section{Analysis} 

In the folder {\tt LEPAnalysis} in the {\tt EPEM} directory are some analysis files
which analyze the events after interfacing the Les Houches file with
\textsf{Herwig++}. \\ The
{\tt .cc} files contain the main programs which provide
histograms for various LEP event shapes. Open the {\tt Makefile} and set {\tt HPDIR} and
{\tt THEPEGDIR} to the folder where you installed \textsf{Herwig++} and \textsf{ThePEG}. Compile the directory by typing the following commands\\
\\
{\tt make} \\
{\tt make install} \\
\\
in the {\tt LEPAnalysis} directory. This recreates the {\tt .so} libraries. You will need
to do this every time to make a change in the analysis files. 

\section{Interfacing with \textsf{Herwig++}}
Having generated the Les Houches file and set up the analysis handlers, the next step is to
run \textsf{Herwig++}. It is assumed that both \textsf{Herwig++} and \textsf{ThePEG} have
already been installed on your system. 

Now go to the directory {\tt MCPWNLO/Herwig} and open the initialization file {\tt
  EPEM.in}. This contains \textsf{Herwig++} user parameters which can be set depending on
which NLO method you are running. The default set-up is for {\tt MC@NLO}. A selection of
these are:\\
\\
{\tt set Reader2:FileName EPEM.dat}: This is the filename for the Les Houches file your
generated file is converted into by the program {\tt EPEM/run\_epem.cxx}. Leave this
as it is! \\
{\tt set Reader2:EBeamA 45.6}: The beam energy in GeV of the electrons. \\
{\tt set Reader2:EBeamB 45.6}: The beam energy in GeV of the positrons.\\
{\tt set LesHouchesHandler:WeightOption NegUnitWeight}: The weight option for the
events. This allows for negative weighted events. \\
{\tt insert SimpleEE:MatrixElements 0 MEee2gZ2qq}: The hard process. Here it's set up for
$Z$/gamma production and decay into $q \bar{q}$ pairs.\\
\\
Next are commands which should be uncommented if running {\tt POWHEG}. If running {\tt MC@NLO}, comment these out. \\
\\
{\tt set /Herwig/Shower/Evolver:HardVetoMode 0}: The veto mode to be applied. For {\tt
  POWHEG}, this should be set to $1$.\\
{\tt set /Herwig/Shower/PartnerFinder:PHPartnerFinder 0}: The partner finder option. This
should be set to $1$ for {\tt POWHEG}.\\
{\tt set /Herwig/Shower/Evolver:HardVetoScaleSource 0}: This reads the maximum $p_T$ for
the veto from {\tt SCALUP} in the Les Houches file. Set this to 1 for {\tt POWHEG}. \\
\\
Next are some tuned parameters to reproduce LEP multiplicities and the best fit to LEP
eventshapes for the {\tt POWHEG} and {\tt MC@NLO} methods.\\
\\
{\tt set /Herwig/Shower/AlphaQCD:AlphaMZ 0.118}: This sets the value of $\alpha_S$ at $M_Z$.\\
{ \tt set /Herwig/Shower/SudakovCommon:cutoffKinScale 2.45*GeV}: This sets the shower
cut-off scale.\\
\\
Having set up the initialization file, run \textsf{Herwig++} by typing the following
commands: \\
\\
{\tt .$\backslash$run\_epem [eventfile] [number of events]} \\
\\
An example of the run
command is:\\
\\
{\tt .$\backslash$run\_epem /usera/seyi/MCPWNLO/EPEM/MCEPEM.dat 10000}\\
\\
where you should replace the eventfile with the path to your Les Houches file.
\\
  At the end of the run, {\tt .top} files will be
  produced containing the histograms booked by the analysis handler. If you have topdraw
  installed, you can convert this to a postscript file by typing the command:\\
\\
{\tt td -dPOSTSCRIPT  *.top}
\section{Further Information}
For further information about {\tt MCPWNLO} contact: {\tt seyi@hep.phy.cam.ac.uk}.
\bibliography{manual}
\bibliographystyle{utphys}
\end{document}
%%% Local Variables: 
%%% mode: latex
%%% TeX-master: t
%%% End: 
